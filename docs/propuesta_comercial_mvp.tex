\documentclass[11pt,a4paper]{article}

% Paquetes esenciales
\usepackage[utf8]{inputenc}
\usepackage[T1]{fontenc}
\usepackage[spanish,es-tabla]{babel}
\usepackage[margin=2.5cm]{geometry}
\usepackage{xcolor}
\usepackage{graphicx}
\usepackage{enumitem}
\usepackage{tcolorbox}
\usepackage{array}
\usepackage{booktabs}
\usepackage{titlesec}
\usepackage{colortbl}
\usepackage{parskip}
\usepackage{hyperref}

% Definición de colores corporativos
\definecolor{primary}{RGB}{0, 75, 128} % Azul institucional
\definecolor{secondary}{RGB}{44, 62, 80} % Gris oscuro
\definecolor{accent}{RGB}{200, 50, 50} % Rojo oscuro para alertas
\definecolor{bglight}{RGB}{245, 247, 250} % Fondo claro para cajas

% Configuración de enlaces
\hypersetup{
    colorlinks=true,
    linkcolor=primary,
    filecolor=primary,
    urlcolor=primary,
    pdftitle={Propuesta Estratégica SaaS Gestión Escolar}
}

% Configuración de Títulos
\titleformat{\section}
  {\color{primary}\Large\bfseries}
  {\thesection}{1em}{}
  [\vspace{-0.5em}\rule{\textwidth}{0.5pt}]

\titleformat{\subsection}
  {\color{secondary}\large\bfseries}
  {\thesubsection}{1em}{}

% Estilos de caja personalizada
\newtcolorbox{infoBox}[1][]{
  colback=bglight,
  colframe=primary,
  fonttitle=\bfseries,
  coltitle=white,
  title=#1,
  arc=4mm,
  boxrule=0.8pt,
  left=10pt, right=10pt, top=10pt, bottom=10pt
}

\begin{document}

% Portada
\begin{titlepage}
    \centering
    \vspace*{2cm}
    
    {\Huge \bfseries \color{primary} Propuesta Estratégica y Cronograma\par}
    \vspace{0.5cm}
    {\Large \bfseries Modelo Comercial (MVP B2B)\par}
    
    \vspace{2cm}
    
    \begin{infoBox}[Detalles del Proyecto]
        \begin{itemize}[label=\textcolor{primary}{\textbullet}, leftmargin=*]
            \item \textbf{Fecha:} 21 de Febrero de 2026
            \item \textbf{Mercado Objetivo:} Bolivia
            \item \textbf{Meta:} Lanzamiento de MVP para la gestión académica, asistencia y reportería ministerial bajo la estricta normativa evaluativa de Bolivia.
        \end{itemize}
    \end{infoBox}
    
    \vfill
    
    {\large Preparado por:\par}
    \vspace{0.2cm}
    {\Large \textbf{Lester Rodriguez}\par}
    {\large \color{secondary} Líder Arquitecto en Venezuela\par}
    
    \vspace{2cm}
\end{titlepage}

\newpage

\section{Plan de Entrega Detallado (4 Semanas)}

El proyecto se segmentará en un plan de acción estricto para certificar cada módulo bajo los estándares de producción de la industria de software.

\subsection*{Semana 1: Arquitectura y Módulo Institucional}
\begin{itemize}[label=\textcolor{primary}{$\blacktriangleright$}]
    \item \textbf{Módulo Empresa:} Configuración del Perfil Escolar (Nombre, Ubicación, tipo de Nivel Educativo y Carga de Logotipo Institucional).
    \item \textbf{Base de Datos y Seguridad:} Generación del esquema Multi-tenant en código (para aislamiento de datos), Módulo de Usuarios, control de Roles y Permisos.
    \item \textbf{Malla Curricular Base:} Creación de Niveles o Grados, habilitación oficial de \textit{``Paralelos''} (Secciones A, B, etc.) y alta del banco de Materias.
\end{itemize}

\subsection*{Semana 2: Filiación y Módulo Académico Operativo}
\begin{itemize}[label=\textcolor{primary}{$\blacktriangleright$}]
    \item \textbf{Módulo Filiativo Estelar:}
    \begin{itemize}
        \item Inscripción (Filiación) de Estudiantes ligándolos directamente a su Paralelo.
        \item Registro de Docentes.
        \item Asignación de Cargas Horarias: Vincular al Docente exclusivamente con su Nivel, Paralelo y Materias permitidas.
    \end{itemize}
    \item \textbf{Módulo Académico (Evaluaciones):}
    \begin{itemize}
        \item Generación de planillas de Carga de Notas, asegurando los topes estipulados de Bolivia: \textbf{Ser (10), Saber (45), Hacer (40) y Autoevaluación (5).}
    \end{itemize}
\end{itemize}

\subsection*{Semana 3: Asistencias y Dashboards Estratégicos}
\begin{itemize}[label=\textcolor{primary}{$\blacktriangleright$}]
    \item \textbf{Asistencia Cotidiana:} Sistema de "1 Clic" para que el docente o secretaría marque la asistencia diaria de la sección/paralelo según la lista de filiados.
    \item \textbf{Dashboard Institucional (Vista Admin/Director):}
    \begin{itemize}
        \item Panel en tiempo real: Total de registrados, Totales de Presentes en el Día y Ausentes de la escuela.
        \item \textbf{Cuadro de Honor Dinámico:} Sistema que computa en vivo al "Mejor Estudiante" de cada Paralelo y al "Genio de Oro" del nivel completo.
    \end{itemize}
    \item \textbf{Centralizador Trimestral:} Tabla de Múltiple Entrada (Materias / Estudiantes) que resume las calificaciones definitivas del período en curso.
\end{itemize}

\subsection*{Semana 4: Certificación, Pruebas y Reporte Ministerial}
\begin{itemize}[label=\textcolor{primary}{$\blacktriangleright$}]
    \item \textbf{Auditoría de Cálculos:} Pruebas unitarias para validar las sumatorias dimensionales.
    \item \textbf{Módulo Certificado:} Preparación, formato y exportación de Boletines individuales que plasmen las notas del estudiante para consumo de padres, e integración sugerida para el formato del Ministerio de Educación.
\end{itemize}

\section{Inversión y Lanzamiento}

\subsection{Fase Core de Implementación (MVP Core)}
\begin{tcolorbox}[colback=primary!5!white,colframe=primary!75!black,title=Inversión de Configuración y Desarrollo Base]
    \centering \Large \textbf{Costo único: \$150 USD}
\end{tcolorbox}
\textbf{Incluye:} Arquitectura B2B programada específicamente para soportar el flujo de notas en Bolivia, auditoría algorítmica y primer despliegue funcional en servidor de producción para el inicio de clases (Shared Hosting / CPanel Básico). Esta inversión fondea de forma temprana y exclusiva la primera arquitectura del código base.

\subsection{Modelo de Negocio (SaaS) y Sociedad}
El verdadero valor del sistema reside en su rentabilidad como servicio escalable (\textit{Software as a Service}). Proponemos un modelo formal de participación recurrente entre las partes basado en licencias institucionales:

\begin{itemize}[label=\textcolor{secondary}{\textbullet}, leftmargin=*]
    \item \textbf{Tarifa MRR (Suscripción por Institución):} \textbf{\$30 USD / mensuales}.
    \item \textbf{Estructura Societaria y División en Tercios (33/33/33):} Para garantizar la salud financiera del proyecto y no afectar el bolsillo de los socios, los ingresos de cada suscripción se dividirán en tres partes iguales (\$10 USD cada una):
    \begin{enumerate}
        \item \textbf{1/3 (Costo Operativo - "La Máquina"):} Fondo intocable destinado automáticamente a cubrir renovaciones de infraestructura técnica inicial (Shared Hosting, MySQL en CPanel y Dominios). El sistema se paga total y absolutamente solo.
        \item \textbf{1/3 (Socio Tecnológico):} Utilidad neta para el Líder Arquitecto basado en \textbf{Venezuela} por el mantenimiento del código, prevención de caídas y mejoras de seguridad.
        \item \textbf{1/3 (Socio Comercial):} Utilidad neta para el enlace e inversor en \textbf{Bolivia} por la prospección, cierre de escuelas y atención al usuario final.
    \end{enumerate}
    \item \textbf{Ajuste Inflacionario Anual:} Para asegurar un crecimiento sostenido que combata la inflación y sume rentabilidad al fondo societario, la tarifa de suscripción tendrá un incremento automático pactado de \textbf{\$5 USD} en cada renovación anual por institución.
\end{itemize}

\begin{infoBox}[Nota Estratégica de Escalabilidad]
Al construir el sistema bajo arquitectura \textit{``Multi-tenant''}, el esfuerzo informático para anexar al segundo, quinto o décimo colegio es nulo. El software se escala hacia un ingreso pasivo y masivo.
\end{infoBox}

\vspace{2cm}

\section*{Firma de Conformidad}

\vspace{2cm}

\noindent
\begin{tabular}{@{}p{0.45\textwidth} p{0.1\textwidth} p{0.45\textwidth}@{}}
\hrulefill & & \hrulefill \\
\textbf{Socio Inversor / Ventas} & & \textbf{Lester Rodriguez} \\
Representante en Bolivia & & Líder Arquitecto en Venezuela \\
\end{tabular}

\end{document}
